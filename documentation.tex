\documentclass[a4paper,11pt]{article}
\usepackage[utf8]{inputenc}
\usepackage[czech]{babel}
\usepackage[margin=2.5cm]{geometry}
\usepackage{hyperref}
\usepackage{enumitem}

% Nastavení hypertextových odkazů
\hypersetup{
    colorlinks=true,
    linkcolor=blue,
    filecolor=magenta,
    urlcolor=cyan,
    pdftitle={Délicious - Dokumentace semestrální práce},
    pdfauthor={Oldřich Jan Švehla}
}

\begin{document}

% Titulní informace
\begin{center}
    {\LARGE\bfseries Délicious - Food Delivery Ordering System}\\[0.5cm]
    {\large Dokumentace semestrální práce}\\[1cm]

    \begin{tabular}{rl}
        \textbf{Jméno:} & Oldřich Jan Švehla \\
        \textbf{Email:} & tmwmf@students.zcu.cz \\
        \textbf{Datum vytvoření:} & 9. prosince 2025 \\
        \textbf{Předmět:} & KIV/WEB (Webové aplikace) \\
        \textbf{Název aplikace:} & Délicious \\
    \end{tabular}
\end{center}

\vspace{1cm}

\section{URL aplikace}
Aplikace běží lokálně na XAMPP serveru: \url{http://localhost/web-foodapp/}

\section{Zadání práce}

\subsection{Oficiální zadání}
\textbf{Objednávkový systém (např. rozvoz potravin)}

Jako příklad lze využít (zjednodušený) web některé z firem pro rozvoz potravin zákazníkům. Role: nepřihlášený uživatel, konzument, dodavatel, administrátor. Nepřihlášený uživatel vidí seznam produktů a může se do systému registrovat jako konzument nebo jako dodavatel (ten by měl být schválen administrátorem). Konzument vidí seznam produktů, které může vkládat do košíku a ten následně objednat. Dále vidí své historické objednávky, včetně produktů a data objednání. Dodavatel může do systému vkládat produkty (a např. může vidět, kolikrát byl který produkt objednán) a vidí nevyřízené objednávky, které může označit za vyřízené. Administrátor spravuje uživatele a může upravit vše potřebné.

\subsection{Realizace}
Aplikace \textbf{Délicious} (slovní hříčka: Delicious + Delivery) je webová aplikace pro rozvoz jídla, která implementuje zadání s následujícími funkcemi:

\begin{itemize}[leftmargin=*]
    \item \textbf{Nepřihlášený uživatel:} může procházet seznam produktů, registrovat se jako konzument nebo dodavatel
    \item \textbf{Konzument:} procházení produktů, přidávání do košíku, vytváření objednávek, zobrazení historických objednávek s produkty a datem
    \item \textbf{Dodavatel:} správa vlastních produktů (vytváření, editace, mazání s nahráváním obrázků), zobrazení objednávek obsahujících jejich produkty
    \item \textbf{Administrátor:} schvalování nových dodavatelů, správa všech uživatelů, produktů a objednávek
    \item Hashování hesel (bcrypt), ochrana proti XSS a SQL injection
    \item Responsivní design pro mobilní zařízení i PC
\end{itemize}

\section{Použité technologie}

\subsection{Backend}
\begin{itemize}[leftmargin=*]
    \item \textbf{PHP 8.2.12} - implementace MVC architektury, business logika v Controllers, datová vrstva\\v Models
    \item \textbf{MySQL/MariaDB} - relační databáze se 4 tabulkami (users, products, orders, order\_items)
    \item \textbf{PDO} - databázová vrstva s prepared statements pro ochranu proti SQL injection
    \item \textbf{Twig 3.22} - template engine pro všechny Views s auto-escapingem (XSS ochrana)
    \item \textbf{Composer} - správa závislostí (Twig)
\end{itemize}

\subsection{Frontend}
\begin{itemize}[leftmargin=*]
    \item \textbf{HTML5 + CSS3} - moderní minimalistický design s custom CSS properties
    \item \textbf{Bootstrap 5.3} - responzivní grid systém, utility třídy, komponenty
    \item \textbf{Bootstrap Icons} - ikonová sada použitá v celé aplikaci
    \item \textbf{JavaScript (Vanilla)} - AJAX operace pro košík, správu produktů (fetch API)
\end{itemize}

\subsection{Vývojové prostředí}
\begin{itemize}[leftmargin=*]
    \item \textbf{XAMPP} - Apache server + MySQL databáze
    \item \textbf{Git/GitHub} - verzování kódu
    \item \textbf{.htaccess} - URL rewriting, směrování všech požadavků do public/index.php
\end{itemize}

\section{Adresářová struktura}

\begin{description}[leftmargin=2cm, style=nextline]
    \item[\texttt{app/Controllers/}] MVC Controllers - obsahuje 8 controllerů (AdminController, CartController, HomeController, LoginController, OrderController, ProductController, RegisterController, SupplierController). Každý má metodu index() pro GET a specifické metody pro POST požadavky.

    \item[\texttt{app/Models/}] MVC Models - obsahuje 4 modely (Database.php pro PDO singleton, User.php, Product.php, Order.php). Zajišťují komunikaci s databází pomocí prepared statements.

    \item[\texttt{app/Views/templates/}] Twig templates - 13 šablon (base.twig jako master layout, home.twig, login.twig, register.twig, products.twig, cart.twig, checkout.twig, orders.twig, order\_detail.twig, supplier.twig, admin\_dashboard.twig, admin\_users.twig, admin\_products.twig, admin\_orders.twig).

    \item[\texttt{app/Helpers/}] Pomocné třídy - TwigHelper.php pro inicializaci Twig s konfigurací (cache, autoescape).

    \item[\texttt{app/autoload.php}] Autoloader pro automatické načítání tříd.

    \item[\texttt{public/index.php}] Jediný vstupní bod aplikace (single entry point). Obsahuje switch-based router, který na základě parametru \texttt{?page=} načítá příslušný controller. Inicializuje session.

    \item[\texttt{public/css/style.css}] Vlastní CSS styly - moderní minimalistický design, responzivní pravidla pro mobilní zařízení.

    \item[\texttt{public/uploads/}] Adresář pro nahrané obrázky produktů.

    \item[\texttt{database/install.sql}] Kompletní instalační SQL skript - vytvoří databázi, všechny tabulky a naplní je testovacími daty.

    \item[\texttt{vendor/}] Composer závislosti (Twig).

    \item[\texttt{.htaccess}] Root htaccess - přesměrování všech požadavků do \texttt{public/} adresáře.
\end{description}

\section{Architektura aplikace}

\subsection{MVC Pattern s Single Entry Point}
Aplikace používá architektonický vzor Model-View-Controller:

\begin{itemize}[leftmargin=*]
    \item \textbf{Router (public/index.php):} Veškerý HTTP provoz směřuje do tohoto souboru díky .htaccess. Switch-case na parametr \texttt{\$\_GET['page']} určuje, který controller se načte. Session je inicializována globálně.

    \item \textbf{Models (app/Models/):}
    \begin{itemize}
        \item \texttt{Database.php} - Singleton pattern pro PDO připojení
        \item \texttt{User.php} - správa uživatelů (registrace, přihlášení, seznam, schvalování)
        \item \texttt{Product.php} - CRUD operace pro produkty (create, read, update, delete)
        \item \texttt{Order.php} - správa objednávek (vytváření včetně položek, seznam, detail, změna stavu)
    \end{itemize}

    \item \textbf{Views (app/Views/templates/):} Twig šablony s auto-escapingem. \texttt{base.twig} je master layout s navbar a bloky (title, content, extra\_js). Ostatní templates z něj dědí. AJAX logika je implementována v block extra\_js.

    \item \textbf{Controllers (app/Controllers/):} Každý má metodu \texttt{index()} pro GET (zobrazení view) a specifické metody pro POST (zpracování formulářů, AJAX). Kontrolují oprávnění na základě \texttt{\$\_SESSION['role']}.
    \begin{itemize}
        \item \texttt{HomeController} - úvodní stránka
        \item \texttt{LoginController} - přihlášení/odhlášení, kontrola schválení dodavatele
        \item \texttt{RegisterController} - registrace s volbou role
        \item \texttt{ProductController} - seznam produktů schválených dodavatelů
        \item \texttt{CartController} - košík v session, AJAX operace (add, update, remove, clear)
        \item \texttt{OrderController} - checkout, vytvoření objednávky, seznam, detail
        \item \texttt{SupplierController} - dashboard dodavatele, CRUD produktů s upload obrázků
        \item \texttt{AdminController} - dashboard, správa uživatelů/produktů/objednávek, schvalování dodavatelů
    \end{itemize}
\end{itemize}

\subsection{Bezpečnost}
\begin{itemize}[leftmargin=*]
    \item \textbf{SQL Injection:} Všechny dotazy používají PDO prepared statements
    \item \textbf{XSS:} Twig autoescape='html' v TwigHelper.php
    \item \textbf{Hesla:} \texttt{password\_hash(PASSWORD\_BCRYPT)}
    \item \textbf{Upload souborů:} Validace typu (JPG, PNG, GIF, WEBP), velikosti (max 5MB), unikátní názvy
    \item \textbf{RBAC:} Kontrola role v session před přístupem k chráněným částem
\end{itemize}

\subsection{Databázový model}
\begin{itemize}[leftmargin=*]
    \item \textbf{users} - PK: user\_id, obsahuje email (UNIQUE), password (bcrypt), jmeno, role (konzument/dodavatel/admin), is\_approved, is\_super\_admin
    \item \textbf{products} - PK: product\_id, FK: supplier\_id -> users, obsahuje name, description, price, image
    \item \textbf{orders} - PK: order\_id, FK: customer\_id -> users, obsahuje customer\_name, email, delivery\_address, phone, note, status, total\_price
    \item \textbf{order\_items} - PK: order\_item\_id, FK: order\_id -> orders, FK: product\_id -> products, obsahuje quantity, price
\end{itemize}

\section{Výchozí uživatelské účty}

Všichni uživatelé mají heslo: \textbf{heslo123}

\subsection{Super Administrátor}
\begin{itemize}[leftmargin=*]
    \item Email: \texttt{superadmin@test.cz}
    \item Heslo: \texttt{heslo123}
    \item Oprávnění: správa všech uživatelů včetně ostatních administrátorů, nelze smazat, nejvyšší oprávnění v systému
\end{itemize}

\subsection{Administrátor}
\begin{itemize}[leftmargin=*]
    \item Email: \texttt{admin@test.cz}
    \item Heslo: \texttt{heslo123}
    \item Oprávnění: správa uživatelů, schvalování dodavatelů, správa všech produktů a objednávek
\end{itemize}

\subsection{Dodavatelé (schválení)}
\begin{itemize}[leftmargin=*]
    \item Email: \texttt{dodavatel@test.cz} - Pizza House (4 produkty)
    \item Email: \texttt{dodavatel2@test.cz} - Burger King (4 produkty)
    \item Heslo: \texttt{heslo123}
    \item Oprávnění: správa vlastních produktů, zobrazení objednávek vlastních produktů
\end{itemize}

\subsection{Dodavatel (neschválený)}
\begin{itemize}[leftmargin=*]
    \item Email: \texttt{dodavatel3@test.cz} - Sushi Bar
    \item Heslo: \texttt{heslo123}
    \item Stav: čeká na schválení administrátorem, nemůže se přihlásit
\end{itemize}

\subsection{Zákazníci}
\begin{itemize}[leftmargin=*]
    \item Email: \texttt{zakaznik@test.cz} - Jan Novák (2 testovací objednávky)
    \item Email: \texttt{zakaznik2@test.cz} - Marie Svobodová (2 testovací objednávky)
    \item Heslo: \texttt{heslo123}
    \item Oprávnění: procházení produktů, košík, vytváření objednávek
\end{itemize}

\end{document}
